\chapter{МЕТОД STRUCTURE FROM MOTION}

\textbf{Structure from Motion} (дословно \quotes{\textit{структура из движения}}, SFM) -- техника построения трёхмерных структур из последовательности двухмерных изображений (фотографий, кадров видео), используемая в области компьютерного зрения. В биологии этот термин описывает феномен, позволяющий человеку восстанавливать трёхмерную структуру окружающего мира по двухмерным проекциям на сетчатку глаза.

\section{Основные тереоретические понятия}

Люди с рождения обладают зрением, что позволяет нам распознавать изображения и объекты на них, сравнивать их между собой, оценивать расстояния и размеры. Всё это человеческий мозг делает бессознательно, автоматически. Однако, для машины изображение — всего лишь закодированные данные, набор нулей и единиц. Одной из основных проблем в сопоставлении изображений является очень большая размерность пространства, которое обладает информацией. Если взять маленькую картинку размером хотя бы $100*100$ пикселей, то уже получим количество информации равное $10^4$ пикселей. Поэтому методы анализа изображения должны быть быстрыми и эффективными.

Существует ряд других проблем и сложностей при попытке интерпретации изображений. Одни и те же объекты на разных изображениях могут очень сильно отличаться. На это влияют такие параметры как точка зрения, освещение (один и тот же объект может быть белым, серым или чёрным в зависимости от освещения), масштаб, деформация и перекрытие объектов, маскировка (слияние с фоном). Поэтому для полноты картины требуется как можно больше различных изображений одного объекта.

\textbf{Так как же машина обретает зрение?} Основная идея состоит в том, чтобы получить какую-то характеристику, которая будет хорошо описывать изображение, легко вычисляться и для которой можно ввести оператор сравнения. Эта \quotes{характеристика} должна быть устойчива к различным преобразованиям (сдвиг, поворот и масштабирование изображения, изменение яркости, изменение положения камеры). Это необходимо для того, чтобы было возможно определить один и тот же объект на изображениях, сделанных с разных углов, расстояний и при разном освещении.

Все эти условия приводят к необходимости выделения на изображении особых, \textit{ключевых точек} (\textbf{key points}). Этот процесс называется \textit{извлечение признаков} (\textbf{feature extraction}). Ключевая точка (локальная особенность) -- это такая особая точка, которая достаточно хорошо отличается от близлежащих точек по какой-то определённой характеристике. Она должна быть сильно \quotes{непохожа} на остальные точки и являться уникальным свойством изображения в своей локальной области. Таким образом, машина может представить изображение как модель, состоящую из особых точек. Например, на изображении человеческого лица функции ключевых точек могут выполнять глаза, уголки губ, кончик носа.

К особым точкам предъявляются следующие требования:
\begin{enumerate}
    \item Повторяемость (Repeatability). Особенность не должна менять свое положение при изменении точки зрения или освещения;
    \item Значимость (Saliency). Каждая особенность должна иметь уникальное описание;
    \item Компактность (Efficiency). Количество особенностей должно быть существенно меньше числа пикселей изображения;
    \item Локальность (Locality). Особенность должна занимать небольшую область изображения, чтобы снизить вероянтность перекрытия другими объектами.
\end{enumerate}

После выделения особых точек компьютеру нужно уметь их сравнивать (отличать друг от друга). Этот процесс называется \textit{сопоставление признаков} (\textbf{feature matching}). Для сравнения удобно использовать \textit{дескрипторы} (\textbf{descriptor} -- \quotes{описатель}). Дескриптор -- своеобразный идентификатор ключевой точки, представляющий её в удобном для сравнения и понятном для машины виде. Дескриптор является вектором, содержащим признаки особой точки. Именно благодаря дескрипторам получается инвариантность относительно преобразований изображений.


\section{Последовательность шагов процесса SFM}

На рисунке \ref{fig:sfm} представлена схема, демонстрирующая процесс восстановления 3D модели поверхности. 

Можно выделить следующие составляющие процесса реконструкции: 
\begin{enumerate}
    \item Выделение ключевых точек и дескрипторов;
    \item Сравнение дескрипторов и нахождение паросочетаний соответствующих друг другу особых точек;
    \item Нахождение геометрического преобразования, которое переводит ключевые точки одного изображения в соответствующие им точки другого изображения;
    \item Позиционирование камер (изображений) и расположение их в трехмерном пространстве.
\end{enumerate}

\begin{figure}[h]
    \centering
    \includegraphics[width=1\textwidth]{sfm.png}
    \caption{\textbf{Последовательность шагов процесса SFM}}
    \label{fig:sfm}
\end{figure}

Далее будут подробнее рассмотрены описанные выше этапы и используемые алгоритмы. Алгоритмы, применяемые на первых двух этапах, называют \textit{алгоритмами, основанными на особых точках} (\textbf{feature-based algorithms}).
