\section{Детектор Харриса}

Детектор Харриса (\textit{Harris Corner Detector}, 1988 год) -- один из первых и самых популярных методов извлечения особенностей. Является детектором углов \hyperref[itm:harris]{[\ref{itm:harris}]}.

Определение углов -- метод используемый в компьютерном зрении для извлечения особенностей. Угол, в общем случае, может быть определен как пересечение двух ребер. Также угол может быть определен как точка, для которой есть два разных доминирующих направления градиента в локальной окрестности точки. Таким образом, угол подходит под многие определения особенности и может быть использован в качестве особой точки. 

Метод основан на основном свойстве угла -- выявлении доминирующих градиентов, что следует из определения угла. Визуально его можно описать как сканирование изображения путём движения окна в локальной области какой-то точки и распознавание изменения яркости при различных сдвигах сразу в нескольких направлениях -- характерная особенность угла.

Пусть $I$ -- это изображение, $(x, y)$ -- точка на нём, являющаяся центром окрестности. Для того чтобы измерить изменение окрестности, локальная область передвигается на $(\delta x, \delta y)$ и разница вычисляется по формуле:

\begin{equation}
    f(\delta x, \delta y) = \sum_{x, y} w(x, y) [I(x + \delta x, y + \delta y) -- I(x, y)]^2,
\end{equation}

\begin{flushleft}
    где $w(x, y)$ -- функция веса для элемента суммы.
\end{flushleft}

Чаще всего функция веса полагается равным нормальному распределению: чем ближе точка к центру окрестности, тем больший вклад она вносит, таким образом отбрасываются граничные значения. Получается что приближённое значение изменения яркости равно:

\begin{equation}
    f(\delta x, \delta y) \approx \begin{bmatrix} \delta x & \delta y \end{bmatrix} M \begin{bmatrix} \delta x \\ \delta y \end{bmatrix},
\end{equation}

\begin{flushleft}
    где матрица $M$ -- интенсивность изображения вдоль осей $x$ и $y$ соответственно.
\end{flushleft}

Матрица $M$ вычисляется по формуле:

\begin{equation}
    M = \sum_{x, y} w(x, y) \begin{bmatrix} I_x^2 & I_x I_y \\ I_x I_y & I_x^2 \end{bmatrix}
\end{equation}

Далее вычисляются собственные значения результирующей матрицы. Потом в зависимости от их значений определяется является ли рассматриваемая точка особой. Если оба значения маленькие -- значит что окрестность не меняется и точка не принадлежит углу. Если одно значение существенно больше другого, значит при движении в одном направлении окрестность сильно меняется, а при движении в другом -- остаётся неизменной. Это значит что точка лежит на границе. В случае, когда оба собственных значения матрицы достаточно велики и примерно одинаковы, можно говорить что точка образует угол.

Алгоритм детектора следующий:
\begin{enumerate}
    \item Вычислить градиент изображения в каждом пикселе;
    \item Вычислить матрицу $M$ для каждого пикселя в локальной окрестности;
    \item Вычислить собственные значения матрицы $M$;
    \item Отбросить точки, которые не проходят пороговые значения;
    \item Найти локальные максимумы функции отклика в заданной окрестности;
    \item Выбрать $N$ самых сильных локальных максимумов.
\end{enumerate}

Детектор Харриса является инвариантным к изменению угла поворота, но не инвариантен к изменению масштаба. Это проблему решают детекторы, которые будут рассмотрены далее.