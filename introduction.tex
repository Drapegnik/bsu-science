\chapter*{ВВЕДЕНИЕ}
\addcontentsline{toc}{chapter}{ВВЕДЕНИЕ}

В настоящее время интенсивно развивается такая область информатики как компьютерное зрение (computer vision). Существует множество алгоритмов по распознаванию и поиску объектов на картинке, сравнения изображений, нахождения геометрического преобразования при помощи которого можно из одного изображения получить другое. Самая популярная библиотека, которая предоставляет реализации основных алгоритмов - решение с открытым исходным кодом \hyperref[itm:opencv]{OpenCV [\ref{itm:opencv}]}.

Алгоритмы компьютерного зрения активно используются в системах управления процессами (промышленные роботы, автономные транспортные средства), системах видеонаблюдения, системах организации информации (индексация баз данных изображений), системах моделирования объектов или окружающей среды (анализ медицинских изображений, топографическое моделирование), системах взаимодействия (устройства ввода для системы человеко-машинного взаимодействия), системы дополненной реальности.

Крупнейшая мировая IT корпорация Google уже сейчас разрабатывает self-driving cars (машины с автопилотом) которые, как предполагается, в будущем изменят существующее представление об автомобилях: человеку вообще не придётся управлять машиной. Это должно снизить число аварий, исключая такую причину дорожно-транспортных происшествий как \quotes{человеческий фактор} и, соответственно, сделать передвижение с помощью автомобиля безопаснее. Самый популярный сервис такси - Uber уже использует машины с автопилотом, что в будущем позволит ещё больше снизить стоимость услуг сокращением траты средств на человеческие ресурсы (компания уже автоматизировала процесс заказа такси, и диспетчеры были заменены мобильным приложением). Американская компания Amazon - крупнейший в мире сервис продаж через интернет - открыла магазин без кассиров, в котором с помощью алгоритмов компьютерного зрения определяется какие товары клиент положил себе в корзину и их стоимость автоматически списывается с карты при выходе из магазина. И это только несколько проектов, а общее количество стартапов в этой области которых увеличивается с каждым днём.

Таким образом компьютерное зрение сейчас - это новая, очень популярная и активно развивающаяся область информатики, используемая всеми лидерами отрасли.

Многие проекты, использующие компьютерное зрение, направлены на автоматизацию рутинной работы, уменьшение человеческого труда. Одна из основных задач - улучшение качества жизни путём высвобождения одного из самых дорогих ресурсов - человеческого времени.

\newpage

\chapter*{АКТУАЛЬНОСТЬ И ПРАКТИЧЕСКАЯ ЗНАЧИМОСТЬ}
\addcontentsline{toc}{chapter}{АКТУАЛЬНОСТЬ И ПРАКТИЧЕСКАЯ ЗНАЧИМОСТЬ}

Как следует из названия БПЛА не имеют пилота, но это не значит, что они не пилотируемы. Управление беспилотником требует специального обучения, сосредоточенности и является очень утомительным для оператора. Основополагающим необходимым условием для работы дрона является наличие GPS сигнала, что делает его очень уязвимым и зависимым от внешних обстоятельств. В отсутствие сигнала системы глобального позиционирования дрон теряет управление.

В связи с этим возникает задача нахождения и использования альтернативных источников навигации. Так как почти каждый современный беспилотник оснащён камерой, то возможно применение алгоритмов компьютерного зрения.

С помощью разработанного алгоритма и программного обеспечения будет возможна навигация дрона, используя только камеру. Дополнительные возможности применения обширны: патрулирование заданной территории и обнаружение новых объектов, не присутствовавших ранее, возвращение в заданную точку в случае потери gps сигнала, слежение за данным объектом, построение 3D карт местности, навигация по заданной цифровой карте.

\newpage
