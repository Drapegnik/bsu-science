\chapter{Введение}

\section{Область применения компьютерного зрения}

В настоящее время интенсивно развивается такая область информатики как компьютерное зрение (computer vision). Существует множество алгоритмов по распознаванию и поиску объектов на картинке, сравнения изображений, определения того, как с помощью геометрический преобразований и/или масштабирования можно из одного изображения получить другое. Самая популярная библиотека, которая предоставляет реализации основных алгоритмов - решение с открытым исходным кодом \hyperref[itm:opencv]{OpenCV [\ref{itm:opencv}]}.

Алгоритмы компьютерного зрения активно используются в системах управления процессами (промышленные роботы, автономные транспортные средства), системах видеонаблюдения, системах организации информации (индексация баз данных изображений), системах моделирования объектов или окружающей среды (анализ медицинских изображений, топографическое моделирование), системах взаимодействия (устройства ввода для системы человеко-машинного взаимодействия), системы дополненной реальности.

Крупнейшая мировая IT корпорация Google разрабатывает self-driving cars (машины с автопилотом) и предполагается, что в будущем человеку вообще не придётся управлять автомобилем. Это должно уменьшить число происшествий исключая "человеческий фактор" и, соответственно, сделать передвижение с помощью автомобиля безопаснее. Самый популярный сервис такси - Uber уже использует машины с автопилотом, что в будущем позволит снизить стоимость услуг сокращением траты средств на человеческие ресурсы (Компания уже уменьшила траты, используя мобильное приложение вместо диспетчеров). Американская компания Amazon открыла магазин без кассиров, в котором с помощью алгоритмов компьютерного зрения определяется какие товары клиент положил себе в корзину и их стоимость автоматически списывается с карты при выходе из магазина.

Таким образом компьютерное зрение, наряду с машинным обучением, является сейчас наиболее новой и активно развивающейся областью информатики, используемой всеми лидерами отрасли. Основное применение компьютерного зрения - уменьшение человеческой работы, высвобождения одного из самых дорогих ресурсов - человеческого времени.

\section{Постановка задачи}

Необходимо разработать и применить алгоритм навигации беспилотного летательного аппарата в условиях отсутствия GPS.

Как следует из названия БПЛА не имеют пилота, но это не значит, что они не пилотируемы. Управление беспилотником требует специального обучения, сосредоточенности и является очень утомительным для оператора. Основополагающим необходимым условиям для работы дрона является наличие GPS сигнала, что делает его очень уязвимым и зависимым от внешних обстоятельств. В отсутствие сигнала системы глобального позиционирования дрон является беспомощным и теряет управление.

В связи с этим возникает задача нахождения и использования альтернативных источников навигации. Так как почти каждый современный беспилотник оснащён камерой возможно использование алгоритмов компьютерного зрения.
 
Задачу можно разбить на этапы:
\begin{enumerate}
    \item Построение 3D карты местности:
        \begin{enumerate}
            \item сбор и подготовка данных;
            \item восстановление модели местности;
            \item извлечение gps координат;
         \end{enumerate}
    \item Разработка алгоритма навигации по существующей карте:
         \begin{enumerate}
            \item нахождение себя по снимку на карте;
            \item определение маршрута;
            \item осуществление навигации;
         \end{enumerate}
    \item Оптимизация алгоритма для возможности построения карты в режиме реального времени на борту БПЛА;
\end{enumerate}

\section{Актуальность и практическая значимость}

С помощью разработанного алгоритма и программного обеспечения возможна навигация дрона используя только камеру как в военных, так и в личных целях. Например: патрулирование заданной территории и выявление появления новых объектов, возвращение домой в случае потери gps сигнала, слежение за данным объектом, навигация по заданной графической точке.
\newpage