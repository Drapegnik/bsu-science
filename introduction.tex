\chapter*{ВВЕДЕНИЕ}
\addcontentsline{toc}{chapter}{ВВЕДЕНИЕ}

В настоящее время интенсивно развивается такая область информатики как компьютерное зрение (computer vision), в которой учёные и инженеры решают задачу достижения высокого уровня распознавания и понимания видео и изображений для компьютера (машины). С технической стороны конечная цель -- автоматизация задач, требующих наличия человеческой зрительной системы. Задачи включают в себя методы для получения, обработки, анализа и интерпретации изображений из реального мира в строго структурированном и понятном для программ виде.

Также существуют такие подзадачи как распознавание и поиск объектов на изображении, сравнение и сопоставление изображений, видео трекинг (слежение), поиск геометрического преобразования, переводящего одно изображение в другое, восстановление (руконструкция) 3D моделей и многие другие.

Алгоритмы компьютерного зрения активно используются в системах управления процессами (промышленные роботы, автономные транспортные средства), системах видеонаблюдения, системах организации информации (индексация баз данных изображений), системах моделирования объектов или окружающей среды (анализ медицинских изображений, топографическое моделирование), системах взаимодействия (устройства ввода для системы человеко-машинного взаимодействия), системы дополненной реальности.

Приложения алгоритмов обширны: от подводных управляемых аппаратов до беспилотных летательных. Маленькие колесные роботы и марсоходы НАСА, применение как в военных целях, так и в повседневных задачах.

Крупнейшая мировая IT корпорация Google уже сейчас разрабатывает self-driving cars (машины с автопилотом) которые, как предполагается, в будущем изменят существующее представление об автомобилях: человеку вообще не придётся управлять машиной. Это должно снизить число аварий, исключая такую причину дорожно-транспортных происшествий как \quotes{человеческий фактор} и, соответственно, сделать передвижение с помощью автомобиля безопаснее.

Самый популярный сервис такси Uber уже использует машины с автопилотом, что в будущем позволит ещё больше снизить стоимость услуг сокращением траты средств на человеческие ресурсы (компания уже автоматизировала процесс заказа такси, и диспетчеры были заменены мобильным приложением).

Американская компания Amazon -- крупнейший в мире сервис продаж через интернет -- открыла магазин без кассиров, в котором с помощью алгоритмов компьютерного зрения определяется какие товары клиент положил себе в корзину и их стоимость автоматически списывается с карты при выходе из магазина. Команда Oculus -- подразделение Facebook -- создаёт очки виртуальной реальности Oculus Rift, которые создатель Facebook Марк Цукенберг считает следующим поколением цифровых устройств, которые должны прийти на смену смартфонам.

Многие проекты, использующие компьютерное зрение, направлены на автоматизацию рутинной работы, уменьшение человеческого труда. Одна из основных задач -- улучшение качества жизни путём высвобождения одного из самых дорогих ресурсов -- человеческого времени.

И это только несколько проектов, а общее количество стартапов в этой области увеличивается с каждым днём. Это ведёт к увеличению решений и библиотек, многие из который публикуются в открытом доступе. Например, одним из используемых мной решений было OpenCV -- самая популярная библиотека с открытым исходным кодом, которая предоставляет реализации основных алгоритмов компьютерного зрения.

Таким образом, компьютерное зрение сейчас -- это новая, очень популярная и активно развивающаяся область информатики, с огромным количеством прорывов и открытий которые происходят прямо сейчас.

