\chapter{ЭКСПЕРИМЕНТЫ С МЕТОДОМ SLAM}

Для проверки возможности использования метода SLAM совместно с методами построения и восстановления карты местности было решено имитировать поток данный с БПЛА с помощью внешней веб-камеры, подключённой к порту usb.

\section{Настройка и калибровка камеры}

Для работы с веб-камерой через ROS требуется использовать пакет \textbf{cv-camera}, который предоставляет узел, взаимодействующий с камерой и принимающий от неё поток видео.

Как и во всех задачах компьютерного зрения, перед началом работы требуется откалибровать камеру, то есть получить её внутренние и внешние параметры. С помощью калибровки получаются такие параметры, как фокусное расстояние, угол наклона, и принципиальная точка (точка соответствующая центру фотографии), которые образуют так называемую матрицу камеры и коэффициенты дисторсии (искажения). 

На рисунке \ref{fig:chess} представлен процесс калибровки камеры. 

\begin{figure}[h]
    \centering
    \includegraphics[width=0.9\textwidth]{images/chess.png}
    \caption{Калибровка камеры с помощью изображения шахматной доски}
    \label{fig:chess}
\end{figure}

Процедура калибровки представляет из себя следующее:
\begin{enumerate}
    \item Выбор предмета с известной геометрией, обычно используется изображение шахматной доски;
    \item Подготовка 30 или более изображений выбранного предмета с разных ракурсов и расстояний;
    \item Определение ключевых точек на полученных фотографиях;
    \item Определение коэффициентов дисторсии через минимизацию ошибки;
    \item Нахождение остальных параметров -- через решение уравнений, полученных путем сопоставления изображений.
\end{enumerate}

Полученные параметры камеры экспортируются в файл и используются для конфигурации SLAM алгоритмов.

\section{Связывание через ROS}

Для связывания компонент приложения вместе и запуска их через ROS будем регистрировать и создавать новые узлы. 

Для начала создадим новый узел, который будет отвечать за веб-камеру. Он будет принимать от неё видео поток и публиковать его в специальный канал \textbf{/cv\_camera/image\_raw}. В качестве альтернативы веб-камеры можно использовать снимки из датасетов, которые рассматривались ранее. Для этого напишем простой узел, в котором можно задать частоту кадров в секунду и директорию, в которой находятся снимки. Также можно указать имя канала, в который узел будет публиковать снимки с заданной частотой. Последний вариант предпочтительнее для отладки, так как не требует подключения веб-камеры и её перемещения в пространстве, что является утомительным занятием. Пример работы представлен на рисунке \ref{fig:ros-slam}.

\begin{figure}[h]
    \centering
    \includegraphics[width=0.83\textwidth]{images/ros-slam.png}
    \caption{Пример работы SLAM с веб-камерой}
    \label{fig:ros-slam}
\end{figure}

Далее SLAM запускается как отдельный узел и подписывается на канал с изображениями. После получения данных он отправляет их процессам позиционирования и построения карты ориентиров. При определении положения камеры SLAM публикует эти данные в следующий канал. На этот канал, в свою очередь, подписано приложение, которое при получении данных с камеры визуализирует их и наносит на карту траекторию движения. После этого полученные от SLAM данные используются для инициализации процесса Structure From Motion, что ускоряет процесс построения плотной карты местности.
