\chapter{РАЗРАБОТКА ПРОГРАММНОГО ОБЕСПЕЧЕНИЯ}

\section{Выбор технологий}

К разрабатываемому приложению были поставлены следующие требования: высокая производительность, удобный и кроссплатформенный пользовательский интерфейс, минимум зависимостей. Для выполнения процесса \textbf{Structure From Motion} была выбрана реализация, от Криса Суини (\textit{Chris Sweeney}) - библиотека проективной геометрии с открытым исходным кодом \hyperref[itm:theia]{Theia [\ref{itm:theia}]}. Кристофер - исследователь Вашингтонского университета, занимается разработками в области компьютерного зрения и виртуальной реальности, имеет степень Ph.D., а также множество научных публикаций. Выбор именно этой библиотеки обусловлен несколькими причинами: легковесность (не имеет зависимостей от больших библиотек, таких как OpenCV или Boost), узкая специализация и направленность на решение конкретной задачи, реализация на С++, очень хорошая и подробная документация. 

Для написания графического пользовательского интерфейса отлично подходил \hyperref[itm:qt]{QT [\ref{itm:qt}]}. QT - это кроссплатформенный инструментарий разработки приложений на языке программирования C++. Qt позволяет запускать написанное с его помощью программное обеспечение в большинстве современных операционных систем (\textit{Windows, macOS, Linux}) путём простой компиляции программы для каждой операционной системы без изменения исходного кода. Также предоставлены обширные инструменты по быстрому и удобному созданию интерфейсов.

\section{Разработка алгоритма поиска}
Итак, после выполнения всех этапов Structure From Motion мы имеем 3d модель - реконструкцию поверхности. Модель представляет из себя набор точек пространства, также мы можем привязать к ним gps-данные. Цель - найти расположение нового снимка, не из исходного датасета, на построенной модели и, в последствии, найти геометрическое преобразование и определить точные координаты из которых был сделан искомый снимок.

Для осуществления поиска по модели вместе с каждой 3d точкой сохраняется набор дескрипторов всех особых точек соответствующих этой, реальной точке. В итоге получается следующий алгоритм:

\begin{enumerate}
    \item на вход поступает очередной снимок;
    \item находим ключевые точки и извлекаем соответствующие им дескрипторы;
    \item сравниваем полученные дескрипторы с сохранёнными в модели;
    \item находим камеру из исходного датасета, для которой имеем наилучшее сопоставление;
    \item находим геометрическое преобразование, с помощью которого искомый снимок проецируется на \quotes{лучшую} камеру;
    \item по известным gps-координатам исходной камеры и геометрического преобразования местоположение искомой камеры.
\end{enumerate}

Также, кроме одной камеры, возможно получение всей области, на которую накладывается искомый снимок.

\section{Обзор приложения}

На рисунке \ref{fig:appa} представлен интерфейс разработанного приложения. Модель - швейцарский карьер построенный на датасете взятом из открытых источников.

\begin{figure}[h]
    \centering
    \includegraphics[width=1\textwidth]{appa.png}
    \caption{Appa - приложение для построения и визуализации 3d моделей, осуществления поиска по ним.}
    \label{fig:appa}
\end{figure}

В приложении реализован следующий функционал:

\begin{itemize}
    \item создание / открытие проекта;
    \item просмотр датасета текущего проекта;
    \item извлечение ключевых точек;
    \item построение модели;
    \item визуализация модели;
    \item поиск по построенной модели.
\end{itemize}

Рассмотрим функционал немного подробнее. При создании проекта надо ввести имя проекта, путь к директории с изображениями и директорию для проекта. В этой директории будет создан конфигурационный файл содержащий всю информацию и с ним и будет ассоциирован проект. При визуализации модели красным отрисовываются положения исходных камер с которых видны ключевые точки. При выборе изображений на боковой панели слева точки выбранного изображения, которые попали в конечную модель подсвечиваются синим (см. рисунок \ref{fig:appa}).

\begin{figure}[h]
    \centering
    \includegraphics[width=0.9\textwidth]{appa-options.png}
    \caption{Существует возможность задать различные параметры построения модели.}
    \label{fig:appa-options}
\end{figure}

При построении модели можно настроить такие параметры как: количество потоков в который будет выполнятся каждая часть процесса Structure From Motiom, тип дескриптора и детектора (в данный момент поддерживаются рассмотренный ранее \hyperref[itm:sift]{SIFT [\ref{itm:sift}]}, а также AKAZE), стратегия сопоставления снимков (Brute Force или Cascade Hashing). Остальные настройки касаются внутренних и внешних параметров камеры. (см. рисунок \ref{fig:appa-options})

После выполнения поиска ключевые точки модели сопоставленные с искомым снимком подсвечиваются красным. Рассматривая производительность: поиск на датасете из $127$ снимков, при извлечении порядка $5000$ ключевых точек на каждом изображении осуществляется, в среднем, за $40-50$ секунд.

\section{Выводы}

В этой главе была представлена проделанная практическая работа. Проанализированы новые технологии и решения. Получен результат работы - рабочее приложение, которое можно дорабатывать и развивать. В планах доведение приложение до дистрибуцируемой версии и распространение в свободном доступе.

Анализируя алгоритм и результаты поиска: итоговое время в разы лучше полученного экспериментально в начале исследований. Но этого всё ещё не достаточно, для стабильной работы в реальном времени на борту беспилотного летательного аппарата. Требуется оптимизация и доработка алгоритма поиска.